\begin{tcolorbox}[enhanced,breakable,
	title=General Information,frame style={color=mycolor}]
    \begin{itemize}
        \item Basis: \href{https://basis.uni-bonn.de/qisserver/rds?state=verpublish\&status=init\&vmfile=no&publishid=252386\&moduleCall=webInfo\&publishConfFile=webInfo&publishSubDir=veranstaltung}{Basis}
        \item Website: \href{https://www.math.uni-bonn.de/~lcote/V3D3\_2024.html}{https://www.math.uni-bonn.de/$\sim$lcote/V3D3\_2024.html}
        \item Time slot(s): \highlight{Tuesday: 14-16} Nussallee Anatomie B and \highlight{Friday: 12-14} GHS
        \item Exams: Tuesday \highlight{11.02.2025, 9-11}, Großer Hörsaal, Wegelerstraße 10 and Friday \highlight{21.03.2025, 9-11}, Großer Hörsaal, Wegelerstraße 10
        \item Deadlines: \highlight{Friday before noon}
    \end{itemize}
\end{tcolorbox}

\section{Organization}

\begin{itemize}\beginlecture{01}{08.10.2024}
    \item Four exercise classes, in the break come to the front and sign up. 
    \item First homework is due this Friday 
    \item Exercise sheets are due on Fridays, every week electronically (groups, at most 2)
    \item No published lecture notes by him!
    \item 5 Minute break right before the full hour
    \item Friday after class for questions
\end{itemize}


\section{Course overview}

He assumes we already know about 
\begin{itemize}
    \item Analysis on \(\R^n\)
    \item Basic point set topology
\end{itemize}

\citation{}
For this class: \dhighlight{smooth manifolds} based on \cite{smooth_manifolds}\marginnote{I would also recommend \cite{differential_geometry} and the notes of 
Gabriel Ong\cite{other_notes_of_F4D1}, which are also based on this course}
\begin{itemize}
    \item Intersection between analysis and topology
    \item Exiting: Connections between those two point of views 
\end{itemize}
\dhighlight{Main topics:}
\begin{enumerate}
    \item[Topic 00:] Topological manifolds 
    \item[Topic 01:] Basic theory of smooth manifolds
    \item[Topic 02:] Vector fields on smooth manifolds
    \item[Topic 03:] Tensor calculus and Stokes' theorem 
    \item[Topic 04:] Lie groups, symplectic and Riemannian geometry     
\end{enumerate}

