\chapter{Vector bundles}

\section{Review of linear algebra}

\subsection{The category of vector spaces}

Fix \(\bK\) a field (for this class we only care about \(\bK=\R\)).

Let \(\text{vec}_\bK\) be the category of finite-dimensional \(\bK\) vector spaces:
\begin{itemize}
    \item Objects are (finite dimensional) \(\bK\) vector spaces 
    \item Morphisms are linear maps.
\end{itemize}

The category \(\text{vec}_\bK\) is \dhighlight{abelian}. In particular 
\begin{itemize}
    \item given \(\psi:V\to W\), there exists \[0\to \ker \phi \hookrightarrow V\stackrel{\psi}{\to} W\]
         \[V\stackrel{\psi}{\to} W\to W/\psi(V)\to 0\] %TODO
    \item \(V,W\), can form \(V\oplus W \equiv V\times W\) \marginnote{We can only form direct sums of finitely many vector spaces, since \(\text{vec}_\bK\) only contain finite dimensional vector spaces}
\end{itemize}

The category \(\text{vec}_\bK\) is symmetric monoidal:
\begin{align*}
    \text{vec}_\bK \times \text{vec}_\bK \to \text{vec}_\bK\\
    (V,W)\mapsto V \otimes_{\bK} W
\end{align*}

Note: % See picture

The category \(\text{vec}_\bK\) admits an anti-involution:
\begin{align*}
    (\cdot)^{\vee}:\text{vec}_\bK&\to \text{vec}_\bK^{\text{op}}\\
    V&\mapsto V^\vee \\
    \psi\in\text{hom}(V,W)&\mapsto \psi^\vee \in \text{hom}(W^\vee,V^\vee) \\
    (\cdot)^{\vee\vee} & \equiv \text{id}
\end{align*}

These notions will have corresponding notions for vector bundles.

\subsection{Tensor products}

Recall that the \dhighlight{tensor product} of vector spaces is characterized by:

\dhighlight{Universal property:} If \(\alpha:V^1\times \dots\times V^k\to W\) is a 
multi-linear map, then there exists 
\begin{center}
    \begin{tikzcd}
        \arrow[d,"\pi"]V^1\times \dots \times V^m \arrow[r,"\alpha"]& W\\
        V^1\otimes \dots \otimes V^m \arrow[ru,"\underline{\alpha}\equiv\alpha"]&
    \end{tikzcd}
\end{center}

\begin{definition*}
    Given a vector space \(V\in \text{vec}_\bK\), a \dhighlight{tensor of type} 
    \((k,l)\in \N\times \N\) is an element of 
    \[T^{k,l}V\coloneqq \underbrace{V\otimes \dots \otimes V}_{k\text{ times}}\otimes \underbrace{V^\vee\otimes \dots \otimes V^\vee}_{l\text{ times}}\]   
\end{definition*}

We write \(T^kV\equiv T^{k,0}V\) and \(T^lV^\vee\equiv T^{0,l}V\)

\begin{remark}
    This is not the same as the decomposition into \((l,k)\) forms in complex linear algebra. 
\end{remark}

A tensor \(\alpha\in T^lV^\vee=(T^l V)^\vee\) defines a map \(T^lV\to \bK\), hence also 
a multi-linear map \(V\times \dots\times V\to \bK\). We denote these maps by \(\alpha\) by abuse of notation.

\begin{definition*}
    A tensor \(\alpha\in T^l V^\vee\) is 
    \begin{itemize}
        \item \dhighlight{alternating}: if the induced map \(\alpha:V^1\times \dots\times V^l\to\bK\) satisfies 
            \[\alpha(x_1,\dots,x_i,\dots,x_j,\dots,x_l)=(-1)^{j-i}\alpha(x_1,\dots,x_j,\dots,x_i,\dots,x_l).\]
        \item \dhighlight{symmetric} if the induced map \(\alpha:V^1\times \dots\times V^l\to\bK\) satisfies 
            \[\alpha(x_1,\dots,x_i,\dots,x_j,\dots,x_l)=\alpha(x_1,\dots,x_j,\dots,x_i,\dots,x_l).\]
    \end{itemize}
    We let \(\Lambda^l(V^\vee)\subset T^l V^\vee\) be the subspace of alternating tensors. We let \(\Sigma^lV^\vee\subset T^l V^\vee\)
    be the subspace of symmetric tensors.
\end{definition*}

\markeol{18}