\chapter{Vector bundles}

\section{Review of linear algebra}

\subsection{The category of vector spaces}

Fix \(\bK\) a field (for this class we only care about \(\bK=\R\)).

Let \(\text{vec}_\bK\) be the category of finite-dimensional \(\bK\) vector spaces:
\begin{itemize}
    \item Objects are (finite dimensional) \(\bK\) vector spaces 
    \item Morphisms are linear maps.
\end{itemize}

The category \(\text{vec}_\bK\) is \dhighlight{abelian}. In particular 
\begin{itemize}
    \item given \(\psi:V\to W\), there exists \[0\to \ker \phi \hookrightarrow V\stackrel{\psi}{\to} W\]
         \[V\stackrel{\psi}{\to} W\to W/\psi(V)\to 0\] %TODO
    \item \(V,W\), can form \(V\oplus W \equiv V\times W\) \marginnote{We can only form direct sums of finitely many vector spaces, since \(\text{vec}_\bK\) only contain finite dimensional vector spaces}
\end{itemize}

The category \(\text{vec}_\bK\) is symmetric monoidal:
\begin{align*}
    \text{vec}_\bK \times \text{vec}_\bK \to \text{vec}_\bK\\
    (V,W)\mapsto V \otimes_{\bK} W
\end{align*}

Note: % See picture

The category \(\text{vec}_\bK\) admits an anti-involution:
\begin{align*}
    (\cdot)^{\vee}:\text{vec}_\bK&\to \text{vec}_\bK^{\text{op}}\\
    V&\mapsto V^\vee \\
    \psi\in\text{hom}(V,W)&\mapsto \psi^\vee \in \text{hom}(W^\vee,V^\vee) \\
    (\cdot)^{\vee\vee} & \equiv \text{id}
\end{align*}

These notions will have corresponding notions for vector bundles.

\subsection{Tensor products}

Recall that the \dhighlight{tensor product} of vector spaces is characterized by:

\dhighlight{Universal property:} If \(\alpha:V^1\times \dots\times V^k\to W\) is a 
multi-linear map, then there exists 
\begin{center}
    \begin{tikzcd}
        \arrow[d,"\pi"]V^1\times \dots \times V^m \arrow[r,"\alpha"]& W\\
        V^1\otimes \dots \otimes V^m \arrow[ru,"\underline{\alpha}\equiv\alpha"]&
    \end{tikzcd}
\end{center}

\begin{definition*}
    Given a vector space \(V\in \text{vec}_\bK\), a \dhighlight{tensor of type} 
    \((k,l)\in \N\times \N\) is an element of 
    \[T^{k,l}V\coloneqq \underbrace{V\otimes \dots \otimes V}_{k\text{ times}}\otimes \underbrace{V^\vee\otimes \dots \otimes V^\vee}_{l\text{ times}}\]   
\end{definition*}

We write \(T^kV\equiv T^{k,0}V\) and \(T^lV^\vee\equiv T^{0,l}V\)

\begin{remark}
    This is not the same as the decomposition into \((l,k)\) forms in complex linear algebra. 
\end{remark}

A tensor \(\alpha\in T^lV^\vee=(T^l V)^\vee\) defines a map \(T^lV\to \bK\), hence also 
a multi-linear map \(V\times \dots\times V\to \bK\). We denote these maps by \(\alpha\) by abuse of notation.

\begin{definition*}
    A tensor \(\alpha\in T^l V^\vee\) is 
    \begin{itemize}
        \item \dhighlight{alternating}: if the induced map \(\alpha:V^1\times \dots\times V^l\to\bK\) satisfies 
            \[\alpha(x_1,\dots,x_i,\dots,x_j,\dots,x_l)=(-1)^{j-i}\alpha(x_1,\dots,x_j,\dots,x_i,\dots,x_l).\]
        \item \dhighlight{symmetric} if the induced map \(\alpha:V^1\times \dots\times V^l\to\bK\) satisfies 
            \[\alpha(x_1,\dots,x_i,\dots,x_j,\dots,x_l)=\alpha(x_1,\dots,x_j,\dots,x_i,\dots,x_l).\]
    \end{itemize}
    We let \(\Lambda^l(V^\vee)\subset T^l V^\vee\) be the subspace of alternating tensors. We let \(\Sigma^lV^\vee\subset T^l V^\vee\)
    be the subspace of symmetric tensors.
\end{definition*}

\markeol{18}

\beginlecture{19}{13.12.2024}

\begin{lemma}\label{lem:8.1}
    \dhighlight{(1):} The inclusion \(\Lambda^k(V^\vee)\subset T^k(V^\vee)\) splits via the map 
    \begin{align*}
        T^k(V^\vee)&\to \Lambda^k(V^\vee)\\
        \alpha&\mapsto \text{alt}(\alpha)\coloneqq \frac{1}{k!}\sum_{\sigma\in S_k}(\text{sgn}(\sigma)) {}^\sigma \alpha
    \end{align*}
    where \[^\sigma \alpha(v_1,\dots,v_k)=\alpha(v_{{\sigma(1)}},\dots,v_{{\sigma(k)}}).\]
    \dhighlight{(2):} The inclusion \(\Sigma^k(V^\vee)\subset T^k(V^\vee)\) also splits via the map 
    \begin{align*}
        T^k(V^\vee)&\to \Sigma^k(V^\vee)\\
        \alpha & \mapsto \text{sym}(\alpha)\coloneqq \frac{1}{k!}\sum_{\sigma\in S_k} {}^\sigma \alpha
    \end{align*}  
\end{lemma}

\begin{proof}\marginnote{We should also check that the maps acutally land in the claimed target set}
    \dhighlight{(1):} We need to show that the composition
    \begin{center}
        \begin{tikzcd}
        \Lambda^k(V^\vee)\arrow[r,"i"] & T^k(V^\vee)\arrow[r,"\text{alt}"] & \Lambda^k(V^\vee)
        \end{tikzcd}
    \end{center}
    is the identity. This is the case because \((\text{sgn}(\sigma)) {}^\sigma\alpha=\alpha\) if \(\alpha \in \Lambda^k(V^\vee)\).

    \dhighlight{(2)} is similar.
\end{proof}

\section{Vector bundles}

\subsection{Basic definitions}

\begin{definition*}
    A (real, smooth) \dhighlight{vector bundle} is a triple \((\pi,E,B)\) where 
    \begin{itemize}
        \item \(E,B\) are manifolds 
        \item \(\pi:E\to B\) is a smooth map
        \item \(E_b=\pi^{-1}(b)\) carries the structure of a real vector space (finite dimensional).\marginnote{If we look at the fibres \dots }
    \end{itemize}
 
    This data must satisfy the following \dhighlight{``local triviality''} condition:

    Given any \(b\in B\), there exists a neighborhood \(b\subset U\) and a diffeomorphism \(\psi:\pi^{-1}(U)\simeq U\times V\)
    such that: \marginnote{\(V\) is some real, finite dimensional vector space}
    \begin{enumerate}
        \item[(i)] \(\pi\circ \psi(x,v)=x\)
        \item[(ii)] \(\psi\restrict{E_b}:E_b\stackrel{\simeq}{\to} \{b\} V\) is an isomorphism of real vector spaces.  
    \end{enumerate}
    
    \(B\) is called the \dhighlight{base}, \(\pi\) the \dhighlight{projection} and \(E\) is called the \dhighlight{total space}.

\end{definition*}

\begin{remark}
    We assume in this course that \(\text{dim}_\bK(E_b)\) is constant (this is automatic if \(B\) is connected).
    Under this convention we can assume that \(V=\R^k\), since every finite dimensional real vector space is non-cannoically isomorphic to \(\R^k\) for some \(k\in \N\) by composing \(\psi\) with \((\text{id},\phi_V)\), where \(\phi_V\) is said isomorphism.
\end{remark}

\begin{figure}[H]\label{fig:8.1}
    \centering
    \includegraphics[width=.7\textwidth]{example-image}
    \caption{Sketch }
\end{figure}
In the homework
\begin{figure}[H]\label{fig:8.2}
    \centering
    \includegraphics[width=.7\textwidth]{example-image}
    \caption{Sketch }
\end{figure}
Rigorously \(E:[0,1]\times \R/(0,v)\sim(1,-v)\). Notice that in this case the local triviality really is local, i.e. we can't take \(U=E\).

\begin{definition*}
    A morphism of (smooth, real) vector bundles \((\pi,E,B)\to (\pi',E',B')\) is the data of smooth maps 
    \(F:E\to E',f:B\to B'\) such that 
    \begin{enumerate}
        \item[(i)] \begin{tikzcd}
            \arrow[d,"\pi"] E\arrow[r,"F"] & E'\arrow[d,"\pi'"]\\
            B \arrow[r,"f"] & B'
        \end{tikzcd} commutes. 
        \item[(ii)] \(F\restrict{E_b}:E_b\to E_{f(b)}'\) is a linear map.
    \end{enumerate}
\end{definition*}

\begin{remark} %TODO: Why?
    \(f\) is determined by \(F\) and the condition that \(F\) sends fibers to fibers.\marginnote{Recoverable through quitioning}
\end{remark}

\dhighlight{Notation:} We write \(E=(\pi,E,B), (\pi,E,B)\stackrel{(F,f)}{\to} (\pi',E',B')\) is written as \(F:E\to E'\).

\begin{definition*}
    Given a vector bundle \(E\), a \dhighlight{sub-bundle} is a vector bundle \(F\) over the same base,
    and a map \(i:F\hookrightarrow E\) covering the identity, such that \(F_b\stackrel{i}{\hookrightarrow}E_b\)
    is injective, i.e. 
    \begin{center}
        \begin{tikzcd}
            \arrow[d]F\arrow[r,hook,"i"] & E \arrow[d,"\pi"]\\
            B\arrow[r,"="] & B
        \end{tikzcd}
    \end{center}
    and \(F_b\hookrightarrow E_b\) is injective.
\end{definition*}

\begin{definition*}
    We let \(\text{vectbund}\) be the category whose objects are (smooth, real) vector bundles. We 
    let \(\text{vectbund}(B)\) be the subcategory on vector bundles over \(B\) with morphisms covering the 
    identity.
\end{definition*}

\begin{remark}
    Can similarly set up a theory of \(C^0,C^k,k\geq 1\) vector bundles over \(\R,\C\). This gives rise to 
    analogous categories. Other would therefore write \(\text{vectbund}^\infty\) for what is here jut called \(\text{vectbund}\). 
\end{remark}

\begin{lemma}[Construction]\label{lem:8.2}
    Let \(f:B\to C\) be a smooth map. Then there is a function \(f^\star:\text{vectbund}(C)\to \text{vectbund}(B)\).
    This is called \dhighlight{pullback}. On objects, we have \(f:\underbrace{E}_{\in \text{vectbund}(C)}\to f^*(E)\coloneqq B\times_C E \to B\).
\end{lemma}
\dhighlight{Note:} Here we use the diagram: 
\begin{center}
    \begin{tikzcd}
        \arrow[d,"\pi_B"]f^*(E)\arrow[r] & E\arrow[d,"\pi"]\\
        B \arrow[r,"f"] & C 
    \end{tikzcd}
\end{center}

\begin{proof}
    Note that \(\pi:E\to C\) is a submersion (always true for vector bundles, follows from the condition \(\pi^{-1}(U)\simeq U\times V\)).
    Hence \(f \pitchfork\pi\) %TOFIX 
    are transverse, and the fiber product exists (in the category \(\maninf\)).
    \begin{itemize}
        \item \(\pi_B^{-1}(b)=\pi^{-1}(f(b))\) endows \(\pi_B^{-1}(b)\) with the structure of a vector space
        \item given \(b\in B\) there exist \(U\ni f(b)\), and \(\psi:\pi^{-1}(U)\sim U\times V\). Hence \(\pi_B^{-1}(U)\simeq \pi^{-1}(U)\times V\), which verifies local triviality.
    \end{itemize}

    There is more to check, but the rest is trivial.
\end{proof}

\subsection{Examples of vector bundles}

\begin{example}
    Let \(B=\{x\}\). Then \(\text{vectbund}(\{x\})\simeq \text{vect}_\R\) by 
    \[\{x\}\times V \stackrel{\pi}{\to} \{x\}\mapsto V\] % TODO other direction
\end{example}

\begin{example}
    Let \(B\) be our favorite smooth manifold. Then there is a unique map \(f:B\to \{x\}\).
    Hence by lemma \ref{lem:8.2}, we have a vector bundle \(f^\star V\). In fact 
    \(f^\star V= B\times V\stackrel{\pi_B}{\to} B\) by \((b,v)\mapsto b\).\marginnote{This is a small exercise}
    Any vector bundle of the form \(f^star V\) is called \dhighlight{trivial}. 
\end{example}

\begin{remark}
    The Möbius bundle is not trivial:
    \(M\stackrel{\pi}{\to} S^1\), if \(M=S^1 \times \R\to S^1\)  \((x,1)\mapsto x\) %TODO: again swap direction
    , then we could define a section \(\sigma:S^1\to M,\pi\circ \sigma=\text{id}\) and \(\sigma\neq 0\).
    Contradiction via the mean value theorem.
\end{remark}

\begin{example}
    Let \(B\) be any manifold. Then \(TB\stackrel{\pi}{\to} B\) is a vector bundle.
    \marginnote{Given a manifold: is the vector bundle trivial? is a nice question to ask (in the exam?)}
\end{example}

\dhighlight{Exercise:} \(T S^1\simeq S^1\times \R\) 

What about \(TS^2\)?

\subsection{Vector bundles from gluing data}

This gives a mechanism to produce many more examples.

\dhighlight{Construction}: % Own box? 
Let \(M\) be a manifold. Let 
\begin{itemize}
    \item \(\{U_\alpha\}_{\alpha\in \cA}\) be a cover of \(M\)
    \item \(\{V_\alpha\}_{\alpha\in \cA}\) be a collection of vector spaces 
    \item \(\varphi_{\alpha\beta}:U_\alpha\cap U_\beta\to \text{GL}(V_\alpha,V_\beta)\simeq \text{GL}(\R^k)\subset \R^{k^2}\) be 
          smooth maps satisfying \[\varphi_{\alpha\gamma}=\varphi_{\beta\gamma}\varphi_{\alpha\beta}\] on  \(U_\alpha\cap U_\beta\cap U_\gamma\) for all \(\alpha,\beta,\gamma\in\cA\). 
          The equation is called \dhighlight{cocycle condition}.
\end{itemize}
We let \begin{equation}
    E\coloneqq \coprod_{\alpha\in\cA} U_\alpha \times V_\alpha /\sim     
\end{equation}
where \(U_{\alpha}\times V_\alpha \ni (x,v)\sim (x,w)\in U_\beta\times V_\beta\)
if \(w=\varphi_{\alpha\beta}(x)(v)\).

We let \(\pi:E\to M\) be the forgetful map \[[(x,v)]\mapsto x.\]

\begin{lemma}\label{lem:8.3}
    \begin{enumerate}
        \item[(a)] \(\pi:E\to M\) is a vector bundle
        \item[(b)] All vector bundles arise in this way.  
    \end{enumerate}
\end{lemma}

\begin{proof}[Proof sketch]
    \dhighlight{(a)} is obvious (once you believe \(\sim\) is an equivalence relation)

    \dhighlight{(b):} Let \(\pi:E\to B\) be a vector bundle, cover 
    \(B\) by \(\{U_\alpha\}\), \(\psi:\pi^{-1}(U_\alpha)\to U_\alpha\times V_\alpha\) and let \(\varphi_{\alpha,\beta}=\psi_\beta\circ \psi_\alpha^{-1}\). \qedhere 
\end{proof}

\markeol{19}

