\chapter{Integration theory}

\section{Orientations of vector spaces}

Let \(V\) be a real vector space of dimension \(n\geq 0\). Then:
\begin{itemize}
    \item \(\Lambda^n V^\star\) is a 1 dimensional vector space 
    \item \(\Lambda^nV^\star\setminus\{0\}\), carries an action of  \(\R_+\coloneqq [0,\infty)\)
          by scaling \((\lambda,\omega)\mapsto \lambda \omega\). The quotient \(\Lambda^n V^\star\setminus\{0\}/\R_+\) is a 
          is a set with two elements, with discrete topology. 
    \item Let \(B=\{v_1,\dots,v_n\},\tilde{B}=\{\tilde{v_1},\dots,\tilde{v_n}\}\). We write \(M_B^{\tilde{B}}\) for the transition matrix 
          i.e. \[v_i=\sum_{j=1}^n (M_B^{\tilde{B}})_ij \tilde{v}_j\]
\end{itemize}

\dhighlight{Exercise:} If \(B^\star,\tilde{B}^\star\) are the dual basis, then \(M_{B^\star}^{\tilde{B}^\star}=(M_B^{\tilde{B}})^\star\).

\begin{definition*}\marginnote{Only for this lecture}
    Given \(V\) as above, we write \(B\sim \tilde{B}\iff \det(M_B^{\tilde{B}})>0\).
\end{definition*}

Clearly this is an equivalence relation.

\begin{lemma}\label{lem:11.1}
    Let \(V\) be a finite dimensional vector space. TFAE:
    \begin{enumerate}
        \item[(i)] an equivalence class \([B]\) of bases for \(V\)
        \item[(ii)] an equivalence class \([B^\star]\) of bases for \(V^\star\)
        \item[(iii)] an element of \(\Lambda^{\text{top}}V^\star\setminus \{0\}\) \marginnote{\(\Lambda^{\text{top}}V\setminus \{0\}=\Lambda^{n}V\setminus \{0\}\), where \(n=\dim V\). He also didn't use a star after \(V\)}.  
    \end{enumerate}
    We call any one of these equivalent pieces of data an \dhighlight{orientation}.
\end{lemma}

\begin{proof}
    \((i)\iff(ii)\) is clear.

    (ii) \(\implies\) (iii). If \(B=\{v_1,\dots,v_n\}\) basis of \(V^\star\), then 
    \(v_1\wedge \dots\wedge v_n\in \Lambda^{\text{top}}V^\star\setminus\{0\}\). If \(\tilde{B}=\{\tilde{v}_1,\dots,\tilde{v}_n\}\)
    is any other basis, then \(\tilde{v}_1\wedge \dots\wedge \tilde{v}_n=\det(M_{B}^{\tilde{B}})v_1\wedge \dots\wedge v_n\).
\end{proof}

\begin{definition*}[Interior multiplication]
    Let \(V\) be a finite-dimensional, \(R\)-vector space. Fix \(v\in V,\omega\in \Lambda^k V^\star\). Then we 
    let \(\iota_v\omega(\cdot,\dots\cdot)\coloneqq \omega(v,\cdot,\dots,\cdot)\in \Lambda^{k-1}V^\star\)
\end{definition*}

\begin{lemma}[Induced orientation]\label{lem:11.2}
    Let \(V\) be a finite-dimensional, \(\R\)-vector space. Let \(j:H\hookrightarrow V\) be a \(n-1\) dimensional 
    subspace. Fix an orientation \(\omega\in \Lambda^{n}V^\star\setminus\{0\}\). Fix \(v\in V\) transverse to \(H\).
    \begin{figure}[H]\label{fig:11.1}
        \centering
        \includegraphics[width=.7\textwidth]{example-image}
        \caption{Sketch 11.1}
    \end{figure}  
    Then:
    \begin{enumerate}
        \item[(i)] \(\iota_v\omega\restrict{H}=j^\star(\iota_v\omega)\in \Lambda^{n-1}H^\star\) is an orientation on \(H\)\marginnote{it is non-zero!}
        \item[(ii)] the orientation from (i) only depends on the connected component of \(V\setminus H\) in which \(v\) is contained. 
    \end{enumerate} 
\end{lemma}

\begin{proof}
    \dhighlight{(i)} need to check, that \(j^\star(\iota_v\omega)\neq 0\).  Pick a basis \(\{y_1,\dots,y_{n-1}\}\) for \(H\). Then 
    \begin{align*}
        (j^\star(\iota_v\omega))(y_1,\dots,y_{n-1})=\omega(v,y_1,\dots,y_{n-1})\neq 0
    \end{align*}
    since \(v,y_1,\dots,y_{n-1}\) is a basis for \(V\).
    
    \dhighlight{(ii)} If \(v,w\) are in the same connected component, then \(v=\lambda w\sum_{i=1}^n a_iy_i\), where 
    \(\lambda>0,a_1,\dots,a_{n-1}\). Then 
    \begin{align*}
        \iota_v\omega(y_1,\dots,y_n)&=\omega(v,y_1,\dots,y_{n-1})\\
        &=\omega(\lambda w+ \sum_{i=1}^{n-1}a_i y_i,y_1,\dots,y_{n-1})\\
        &=\omega(\lambda w,y_1,\dots,y_{n-1})+\underbrace{\omega(\sum_{i=1}^{n-1}a_iy_i,y_1,\dots,y_{n-1})}_{=0}\\
        &=\lambda\omega(w,y_1,\dots,y_{n-1})\\
        &=\lambda\iota_w(y_1,\dots,y_{n-1})\\
        \implies \iota_v\omega&=\lambda\iota_w\omega
    \end{align*}
\end{proof}

\section{Orientations of smooth manifolds}

\begin{definition*}
    Let \(M\) be a smooth manifold.  An \dhighlight{orientation} is a smooth section 
    \(\sigma\in\Gamma^1(\Lambda^{\text{top}}T^\star M)=\Omega^{\text{top}}(M)\), which is non-vanishing, modulo the 
    equivalence relation \(\sigma\sim\sigma'\iff \exists f:M\to\R_+,\sigma'=f\sigma\).
    
    We call the data of a manifold + an orientation an \dhighlight{oriented manifold}.
\end{definition*}

\begin{remark}[for topology enthusiast]\marginnote{non examable, proved by approximation}
    The following definitions are equivalent to the one above:
    \begin{enumerate}
        \item[(i)] a continuous \(\sigma\) of \(\Lambda^{\text{top}}T^\star M\), 
                   which is non-vanishing, modulo \(\sigma\sim\sigma'\iff \exists f:M\to\R_+\), continuous, \(\sigma'=f\sigma\). 
        \item[(ii)] A section of the fiber bundle \((\Lambda^{\text{top}}T^\star M\setminus 0_M)/\R_+\stackrel{\pi}{\to}M\ni p\)
                    \(\pi^{-1}(p)=(\Lambda^{\text{top}}T_p^\star M\setminus 0_M)/\R_+\)            
    \end{enumerate}
\end{remark}

\begin{example}
    \(\R^n,\omega_0\coloneqq dx_1\wedge\dots\wedge dx_n\in\Omega^n(\R^n)\). We call 
    \(\omega_0\) the \dhighlight{canonical} orientation.
\end{example}

\begin{example}[Non-example / Warning]
    Not all manifolds admit an orientation! E.g. \(\R\bP^{2n}\) are non-orientable, similarly the Möbius band.
\end{example}

If we have a manifold, the \(TM\) is also a manifold and it is always orientable!

\begin{lemma}\label{lem:11.3}
    Let \(M\) be a manifold and let \(j:H\hookrightarrow M\) be a codimension \(1\) submanifold. Fix an orientation 
    \(\omega\) on \(M\). Let \(V\) be a vector field along \(H\), which is transverse to \(H\), i.e. a section \(V\) 
    \((\underbrace{j^\star TM}_{\equiv TM\restrict{H}}\to H)\) s.t. \(\sigma_p\perp T_p H, T_pH\subset T_p M\).
    Then 
    \begin{enumerate}
        \item[(i)] \(j^\star\iota_V\omega \in \Omega^{\text{top}}\) is an orientation
        \item[(ii)] if \(W\) is a vector field along \(H\), transverse to \(H\), such that \(V_p,W_p\) lie in the same connected component
                    of \(T_pM\setminus T_pH\) for all \(p\in H\). Then \(j^\star(\iota_V\omega)=j^\star(\iota_W\omega)\).   
    \end{enumerate}
    We call this orientation the \dhighlight{induced orientation} (depends on \(H\subset M,\omega,V\)).
\end{lemma}

\begin{proof}
    Immediate corollary of lemma \ref{lem:11.2}.
\end{proof}

\begin{definition*}
    Let \(M\) be a manifold with boundary. A vector \(v\in T_pM,p\in\partial M\subset M\) is said to be 
    \dhighlight{inward-pointing} if there exists a curve \(\gamma:[0,\epsilon)\to M,\dot{\gamma}(0)=v\notin T_p \partial M\). We say 
    \(w\in T_pM\) is \dhighlight{outward pointing} if \(-w\) is inward-pointing. 
\end{definition*}

\dhighlight{Observe:} Any positive linear combination of inward-pointing vectors is inward-pointing.
If we have \(a_1,\dots,a_n\geq 0,\sum a_i=1\), \(v_1,\dots,v_n\) inward-pointing at \(p\), then so is \(\sum_{a_i}v_i\).


\begin{lemma}\label{lem:11.4}
    Let \(M\) be a manifold with non-empty boundary.
    \begin{enumerate}
        \item[(i)] There exists an inward pointing vector field (also outward-pointing vector fields)
        \item[(ii)] If \(\omega\in\Lambda^{\text{top}}(M)\) orientation on \(M\), \(Z\) any outward-pointing vector field, then 
                    \(j^\star(\iota_Z\omega)\in \Omega^{\text{top}}\), \(j:\partial M\hookrightarrow M\), is an orientation. 
    \end{enumerate}
    We call \(j^\star(\iota_Z\omega)\in\Omega^{\text{top}}(\partial M)\) the \dhighlight{induced / Stokes orientation} on \(\partial M\). It does not depend on the choice of outward pointing vector field.
\end{lemma}


\begin{proof}
    \dhighlight{(ii):} Immediate consequence of lemma \ref{lem:11.3}.

    \dhighlight{(i):} We seek \(Z\) a section of \((\underbrace{j^\star TM}_{=\partial M\substack{\times\\M} TM}\to \partial M)\)
    s.t. \(\forall p\in \partial M, Z_p\in T_p\partial M\subset T_p M\), \(Z_p\) lies in the \highlight{canonical} component of 
    \(T_p M \setminus T_p \partial M\). Choose a covering of \(\partial M\) by charts \((U_\alpha,\varphi_\alpha),\varphi_\alpha:U_\alpha\to\bH^n\).
    Choose a subordinate partition of unity \(\{\eta_\alpha\}_{\alpha}\).

    Let \(Z_0\in\Gamma(\bH^n),Z_0=\partial_{x_n}\)
    \begin{figure}[H]\label{fig:11.2}
        \centering
        \includegraphics[width=.7\textwidth]{example-image}
        \caption{Sketch 11.2}
    \end{figure}
    Let \(Z=\sum_{\alpha}\eta_\alpha(d\varphi_{\alpha}^{-1}(Z_0))\) this works 
    because of the previous observation about positive combinations of inward pointing vectors.    
\end{proof}

\begin{example}
    Let \(B^{n+1}(1)\subset \R^{n+1},\partial B^{n+1}(1)=S^n\).
    We have \(Z=x_1\partial_{x_1}+\dots+x_n\partial_{x_n}\). Clearly \(\forall p\in S^n, Z_p\perp T_p S^n\)
    \begin{figure}[H]\label{fig:11.3}
        \centering
        \includegraphics[width=.7\textwidth]{example-image}
        \caption{Sketch 11.3}
    \end{figure}
    \(\implies\) By lemma \ref{lem:11.4}, \(i_z \omega_0\) on \(S^n\).
\end{example}

\markeol{25}
